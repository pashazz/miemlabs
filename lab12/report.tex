\documentclass[a4paper,12pt,fleqn,twoside]{scrartcl}
%%% fleqn - все align'ы находятся слева

%%% Математика
\usepackage{amsmath,amsfonts,amssymb,amsthm} % AMS
\usepackage{mathtools} % Еще AMS
\usepackage{mathtext} % Русские буквы в фомулах
\usepackage{textcomp} % Чтобы в формулах можно было русские буквы
%\usepackage{minted}
\usepackage{listings}

\lstdefinestyle{myc}{
  belowcaptionskip=1\baselineskip,
  breaklines=true,
  frame=L,
  xleftmargin=\parindent,
  language=C,
  alsolanguage=[x86masm]Assembler,
  showstringspaces=false,
  basicstyle=\footnotesize\ttfamily,
  keywordstyle=\bfseries\color{green!40!black},
  commentstyle=\itshape\color{purple!40!black},
  identifierstyle=\color{blue},
  stringstyle=\color{orange},
  numbers=left,
  numberstyle=\ttfamily
}
\usepackage{polyglossia}   %% загружает пакет многоязыковой вёрстки
\setdefaultlanguage[spelling=modern]{russian}  %% устанавливает главный язык документа
\defaultfontfeatures{Ligatures={TeX},Renderer=Basic}  %% свойства шрифтов по умолчанию

\setmainfont{Times New Roman} %% задаёт основной шрифт документа
\setmonofont{Ubuntu Mono}
\setsansfont{Arial}


\usepackage{geometry}
\geometry{top=25mm}
\geometry{bottom=35mm}
\geometry{left=20mm}
\geometry{right=20mm}
\pagestyle{empty}

%%% Таблицы
\usepackage{multirow}
\usepackage{array}
\usepackage{tabularx}




%% параметры абзаца
\parindent = 0pt
\begin{document}
\begin{center}

\large
\bfseries
Правительство Российской Федерации
\bigskip


Федеральное государственное автономное образовательное учреждение высшего профессионального образования\\
<<Национальный исследовательский университет\\
<<Высшая школа экономики>>\\
\bigskip
\mdseries

Московский институт электроники и математики Национального исследовательского университета <<Высшая школа экономики>>
\vspace{4ex}

Факультет прикладной математики и кибернетики
\vfill

\bfseries
\sffamily
О\,Т\,Ч\,Е\,Т\,\\
\bigskip

По лабораторной работе № 12\\
\bigskip
По курсу <<Программирование>>
\rmfamily \mdseries

\vfill

\small
\extrarowheight=3pt
\begin{tabular}{|l|l|l|l|}
\hline
ФИО студента & Номер группы & Дата & Баллы \\\hline
\multirow{4}{*}{Борисов Павел Геннадьевич} & \multirow{4}{*}{ПИ-11} & \phantom{22.22.2222} & \phantom{10} \\\cline{3-4}
&&&\\\cline{3-4}
&&&\\\cline{3-4}
&&&\\\hline

\end{tabular}
\vfill


\bfseries \normalsize Москва - 2014 г.

\end{center}

\newpage
\section*{Постановка задания}
Задать многочлен от Х односвязным списком. Элемент списка содержит неотрицательный целочисленный показатель степени Х и ненулевой коэффициент при этой степени (в списке не должно быть элементов с одинаковыми степенями). Составить программу, включающую помимо указанных в задании функций, функции создания и вывода списка на экран. Список или списки должны отображаться на экране до обработки и после.

\subsection*{Вариант 3}
Написать функцию удаления коэффициента из представления многочлена (всех элементов, имеющих заданный коэффициент при разных степенях)

\subsection*{Makefile}
\lstinputlisting[style=myc,texcl=true]{Makefile}

\subsection*{linkedlist.c}
\lstinputlisting[style=myc,texcl=true]{linkedlist.c}

\subsection*{linkedlist.h}
\lstinputlisting[style=myc,texcl=true]{linkedlist.h}
\subsection{main.c}
\lstinputlisting[style=myc,texcl=true]{main.c}
\section*{Тесты}
\small
\begin{verbatim}
pasha@primum ~/projects/miem/1/4/12 (git)-[master] % ./lab12
Enter the degree of polynomial (negative to stop): 3
Enter the coefficient[3] (0 to stop):4
Enter the degree of polynomial (negative to stop): 2
Enter the coefficient[2] (0 to stop):4
Enter the degree of polynomial (negative to stop): 1
Enter the coefficient[1] (0 to stop):-5
Enter the degree of polynomial (negative to stop): 0
Enter the coefficient[0] (0 to stop):1
Enter the degree of polynomial (negative to stop): -1
Input: 4x^3 +4x^2 -5x + 1
Enter the coefficient for removing: 4
Result: -5x + 1



 ./lab12
Enter the degree of polynomial (negative to stop): 5
Enter the coefficient[5] (0 to stop):1
Enter the degree of polynomial (negative to stop): 3
Enter the coefficient[3] (0 to stop):2
Enter the degree of polynomial (negative to stop): 6
Enter the coefficient[6] (0 to stop):-2
Enter the degree of polynomial (negative to stop): -1
Input: x^5 +2x^3 -2x^6
Enter the coefficient for removing: 2
Result: x^5 -2x^6

\end{verbatim}

% Code inclusion
% C: \lstinputlisting[style=myc,texcl=true]{file}
% asm: \lstinputlisting[style=myasm]{file}
\end{document}