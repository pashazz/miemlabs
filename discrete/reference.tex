\documentclass[12pt,a4paper, fleqn]{scrartcl} % Класс печатного документа.
% ljhgyfdjgv;jhh

%%% Математика
\usepackage{amsmath,amsfonts,amssymb,amsthm} % AMS
\usepackage{mathtools} % Еще AMS
\usepackage{mathtext} % Русские буквы в фомулах
\usepackage{marvosym}%%% Номера только там, где нужно
% \usepackage{autonum}

%%% Формула в рамке: \boxed{}

%%% Русский язык в тексте
\usepackage[russian]{babel} % Поддержка русского языка.
\usepackage[X2,T2A]{fontenc} % Кодировки
\usepackage[utf8]{inputenc} % Кодировка TeX-файла
\usepackage{cmap} % Поддержка русского в PDF
\usepackage{autonum}

\newcommand*{\hm}[1]{#1\nobreak\discretionary{}% Перенос знаков в формулах
{\hbox{$\mathsurround=0pt #1$}}{}}

%%% Шрифты
%\usepackage{euler}

\usepackage{concrete}

\begin{document}
\title{Combinatorics}
\maketitle
\section{Основы}
\label{sec:basics}

\begin{align}
&  X = \{x_1, x_2, \dots, x_n\} & |X| = n\\
&(x_{i_1}, x_{i_2}, \dots, x_{i_r})\text{ --- выборка длины } r\\
&(n, r) \text{ --- выборка}
\end{align}

\section{Выборки}

\begin{itemize}
\item в которых порядок существенен: \textit{перестановки}: $P_n^r = (x_{i_1}, x_{i_2}, ..., x_{i_r})$
  \\Свойства:
\begin{enumerate}
\item $C_n^r = \frac{P_n^r}{r!}$
\end{enumerate}
\item в которых порядок несущественен: \textit{сочетания}: $C_n^r = [x_{i_1}, x_{i_2}, ..., x_{i_r}]$
\\Свойства:
\begin{enumerate}
\item $C_n^r=C_n^{n-r}$
\item $C_{m+n}^n = C_{m+n}^m$
\item $2^n = \sum_{k=0}^n C_n^k $
\item $C_n^k = C_{n-1}^{k+1} + C_{n-1}^k$
\end{enumerate}
\end{itemize}

Над выборками с повторениями ставится $\overline{\text{крышка}}$

\section{Формулы}
{\scriptsize\slshape  \Info пор --- порядок важен; повт --- допускаются ли повторения; $n$ --- количество элементов в множестве; $r$ --- количество элементов в выборке}\\

\begin{tabular}{|c|c|c|c|c|}
  \hline
  Название & пор & повт & практическая формула & теоретическая формула \\\hline
  Перестановки & \Checkedbox &  \CrossedBox & $P_n^r = n(n-1) \cdot \ldots\cdot (n-r+1) $ &


  $ P_n^r = \frac{n!}{(n-r)!}$\\\hline
  Перестановки & \Checkedbox& \Checkedbox & $\overline{P_n^r} = n^r $& $\overline{P_n^r} = n^r $\\\hline
  Сочетания & \CrossedBox & \CrossedBox & $C_n^r = \frac{n(n-1) \cdot \ldots \cdot (n-r+1)}{r!}$&
  $C_n^r = \frac{n!}{(n-r)!r!}$\\\hline
  Сочетания & \CrossedBox & \Checkedbox & $\overline{C_n^r} = C_{n+r-1}^r$& она же \\\hline
\end{tabular}

\section{Правила}
\label{sec:rules}
\subsection{Правило суммы}
\begin{align}
  |X| &= m\\
  |Y| &= n\\
  X \cap Y &= \emptyset \Rightarrow \\
\Rightarrow &|X \cup Y| = m+n
\end{align}

\subsection{Правило произведения}
\begin{align}
& X \times Y = \{(x;y): \forall x \in X, \forall Y \in Y\} \\
&  |X| = m & |Y| = n \\
& |X \times Y| = mn
\end{align}

\end{document}
